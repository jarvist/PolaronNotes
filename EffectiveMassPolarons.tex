\documentclass[twocolumn,11pt]{article}
\usepackage[utf8]{inputenc} 
    % I've locked the doors, but just in case some unicode gets in
\usepackage{hyperref} % URLs

\title{Effective mass polarons}
\author{Jarvist Moore Frost}
\date{December 2017}

\begin{document}

\maketitle

\section{These notes}

Feynman's 1955 paper 'Slow electrons in polar crystals'\cite{Feynman1955}
introduces his path-integral solution to the polaron problem. 
This solves the Fr\"ohlich Hamiltonian, which considers effective-mass
electrons interacting via polar vibrations induced in the lattice. 

Schultz\cite{Fowler2005} received his thesis, "Electron-Lattice Interactions in
Polar Crystals" in 1956. 
A two-year postdoctoral position in Birmingham working with Rudolf Peierls led
to his 1959 Physical Review, 
"Slow Electrons in Polar Crystals: Self-Energy, Mass, and
Mobility"\cite{Schultz1959}.  
These notes are chiefly concerned with the early part of this paper, where the
Feynman theory is 'fleshed out' with more experimental observables. 

Implementing numerical methods is far easier today than it would have been for
Schultz (or rather, for Hannah Landsman acknowledged 'for most of the numerical
work') with the 'Whirlwind Digital Computer'\cite{WhirlwindWikipedia}, a very early
vacuum tube digital computer built at MIT.  
These notes were developed having read the original literature, in order to
develop a modern set of codes to implement the methods, and so enable
calculation of polaron states for arbitrary materials. 
The codes are developed in the Julia language\cite{Julia}. 
This language is high level and mathematically expressive, as such these notes
are an adjunct to reading the program source codes directly.  
These are available online at
\url{https://github.com/jarvist/PolaronMobility.jl}.

\section{Schultz's polaron radii}

Schultz1959\cite{Schultz1959}  defines the
Feynman polaron radius as the standard deviation of the Gaussian Wavefunction. 
This Gaussian is fully specified by the Simple Harmonic Oscillator parameters
for the polaron system, which in turn are provided by Feynman's variational
solution (Feynman1955\cite{Feynman1955}) to the
Fr\"ohlich polaron problem.

The reduced mass is defined in terms of the internal polaron parameters $v$ and
$w$,

\begin{equation}
	\mu=m(v^2-w^2)/v^2
\end{equation}

Here $m$ is the band effective-mass of the electron (hole). 

The (Gaussian) wavefunction is fully specified (plotted above),

\begin{equation}
	\psi(r)=(\mu v/\pi)^{\frac{3}{4}} exp(-\frac{\mu v r^2}{2}).
\end{equation}

And from the standard deviation of this wavefunction, a
radius is defined:

\begin{equation}
    r_f \equiv (<\rho^2>)^{\frac{1}{2}} = (3/2\mu v)^{\frac{1}{2}}
    \quad \mathrm{(2.4, Schultz)}
\end{equation}    

Here $\rho$ is the density of the wavefunction, $\mu$ is the reduced
(effective) mass of the electron and interacting phonon-cloud, while
$v$ and \(w\) are internal polaron parameters characterising the
harmonic motion of the polaron. The units of $v$ and \(w\) are
$\hbar\omega$.

Schultz: 'Using the weak-$\alpha$ coupling expansions given by Feynman for $w$
and $v$, we find'

\begin{equation}
    r_f (\alpha\rightarrow 0) \approx (3/0.44\alpha)^{\frac{1}{2}} (2m\omega)^{-\frac{1}{2}}
\end{equation}

To understand where this magic number (0.44) comes from, one must return
to Feynman1955 and repeat the derivation used to get the
small-$\alpha$ expansion for energy. 
This is the part around Eqn 35 to 36 in Feynman 1955.

Finding the term $2\mu v$ is our aim.

Following Feynman, assuming $\alpha$ is small, $v=(1+\epsilon)w$.
You can then go back to the energy integral Feynman(31), substituting
your $v$ expansion in. You can expand the square-root on on the
denominator, by pulling out terms of $w$ to simplify, before taking a
term of $\tau^{\frac{1}{2}}$ out to place the root in the form of
$(1+x)^n$. You can then use a binomial expansion for this square root,
and keep only zeroth and first order in $\epsilon$ terms. The
resulting integral in the total energy has two parts, the zeroth-order
contribution integrating to 1, and the linear in $\epsilon$ term (P)
apparently analytic and provided by Feynman (35),

\begin{equation}
    2w^{-1} [ (1+w)^{\frac{1}{2}} - 1] = P .
\end{equation}

This makes the total energy,

\begin{equation}
    E=\frac{3}{4v}(v-w)^2-A 
    \quad \mathrm{, (Feynman33)}
\end{equation}

which with our linear-in-$\epsilon$ form of the integral $A$,

\begin{equation}
    E = \frac{3}{4v}(v-w)^2 - \alpha \frac{v}{w} {[}1-P{]}
\end{equation}

Again, substituting $v=(1+\epsilon)w$,

\begin{equation}
    E=\frac{3}{4}\frac{((1+\epsilon)w - w)^2}{(1+\epsilon)w}
       - \alpha\frac{(1+\epsilon)w}{w} [1-P], 
\end{equation}

and discarding high order (in $\epsilon$) terms,

\begin{equation}
    E = \frac{3}{4} w\epsilon^2 -\alpha -\alpha\epsilon[1-P],
\end{equation}

as given by Feynman.

If we reorder terms,

\begin{equation}
    E = \frac{3}{4} w\epsilon^2 -\alpha\epsilon[1-P]-(\alpha),
\end{equation}

it becomes clear the total energy has a quadratic ($E =a\epsilon^2 + b\epsilon +c $) form in $\epsilon$.

We will want a variational (minimum energy) solution. This found at the
minimum, $x=\frac{-b}{2a}$ in the standard quadratic formula, and so,

\begin{equation}
    \epsilon = \frac{2}{3} \alpha \frac{(1-P)}{w}.
\end{equation}

This is as given by Feynman, valid for small-$\alpha$ only, due to the various
occasions where $\epsilon$ is taken as small.

We can now return to the key $2\mu v$ term.

Expanding this identity with our definition of $\mu$,

\begin{equation}
    2\mu v = 2m_e \frac{v^2-w^2}{v}.
\end{equation}

Substituting in $v=(1+\epsilon)w$, following through the algebra, and
as epsilon is small approximating $\frac{1}{1+\epsilon}=\frac{1}{1}$,
we eventually have

\begin{equation}
    2\mu v = 2m_e[2w\epsilon].
\end{equation}

Substituting in our form for $\epsilon$, by simple algebraic
rearrangement we get

\begin{equation}
    v=(1+\frac{2\alpha(1-P)}{3w})w = w + \frac{2}{3} \alpha (1-P)
\end{equation}

Feynman states that `the variational solution is least for $w=3$', and
goes on to state that the energy correction is anyway insensitive to the
choice of $w$.

If $w=3$, \(P=\frac{2}{3}\), and we have that,

\begin{equation}
    v=3+\frac{2}{9}\alpha .
\end{equation}

Thus, $2\mu v = 2m_e \frac{4}{9}\alpha = 2m_e 0.\overline{4} \alpha$.

Schultz1959 Eqn. 2.5a approximates this as 

\begin{equation}
    2\mu v = 2m_e 0.44 . 
\end{equation}

It is not obvious whether this arises as a printing error, or was an
intentional (1\% error) approximation.

\bibliography{EffectiveMassPolarons}
\bibliographystyle{unsrturl} %plainurl} %unsrt}

\end{document}

